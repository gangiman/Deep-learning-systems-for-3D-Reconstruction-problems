\section{Introduction}

For computer vision systems a precise and robust operations in real environments is only possible by harnessing information from 3D data. To achieve this we need to overcome some challenges of this kind of problems.

Data, received from such devices as $2.5D$ scanners, is often given in the form of noisy meshes or point clouds, which is not the best fit for new kinds of models such as Convolutional Neural Networks (CNN's) \cite{lecun1995convolutional}.

In the current state-of-the-art systems Convolutional Neural Networks are widely used, their effectiveness at processing 2D images is also suggestive of their efficacy to process 3D objects if presented in the form of several rendered views of the object. For example on one ModelNet40 \cite{wu20153d} benchmark three recent papers, based on this idea, showed incremental improvements in recognition performance \cite{su15mvcnn,johns2016pairwise,hegde2016fusionnet}. However, it can be argued that high performance is predicated by the usage of CNNs pre-trained on ImageNet \cite{deng2009imagenet}.

Voxel representation of 3D shapes (i.e. a shape is represented as a three-dimensional grid, where occupied cells are binary values) are compatible with ConvNets input layers but create a number of difficulties.
Adding a third spatial dimension in the input grid correspondingly increases computational costs. Number of cells scales as a power of three w.r.t. the resolution of the voxel grid. Low resolution grids make it difficult to differentiate between similar shapes, and lose some of the fine details available in 2D renderings of equivalent resolution.

Some 3D Dense Convolutional Networks have been evaluated on the ModelNet40 benchmark \cite{maturana2015voxnet,sedaghat2016orientation,wu20153d,brock2016generative}, but they still not perform as well as their multi-rendering 2D counterparts.

At the same time using Modified Spatially Sparse Neural Networks algorithms \cite{graham2014spatially} to process data we are able to have reasonable training and inference time even with input resolution up to $100^3$ voxels.

In this work, we present Sparse 3D Deep Convolutional Neural Networks and explore their ability to perform large-scale shape retrieval on the popular benchmark ModelNet40 \cite{wu20153d} depending on an input resolution and a network architecture.
% and ShapeNetCore55 \cite{chang2015shapenet},

Sparse 3D CNNs are able to generate relevant features for retrieval analogously to 2D extractors. To have a system that uses many 2D rendered projections for inference is computationally very costly, especially for the task of Large Scale 3D Shape Retrieval. In this paper we present some preliminary results of our attempt to find out if the resolution of an object Voxelization impacts on descriptive feature extraction as measured by the retrieval performance on a sufficiently big dataset. Also we demonstrate ability of Sparse 3D CNNs to perform metric learning in the triplet loss setup. Lastly we train our model to perform classification on the ModelNet40 benchmark.

In Section~\ref{sec:2} we formulate the problem in more detail and discuss latest relevant methods.
In Section~\ref{sec:3} we describe our approach to neural networks that helps us to solve the problem posed in Section~\ref{sec:2}.
In Section~\ref{sec:4} we document conditions of computational experiments we performed.
In Section~\ref{sec:5} we discuss results and make conclusions about our approach to the problem.

% Performing triplet metric learning with 3D CNNs was done for ranking architectural 3D models by style\cite{Lim:2016:StyleLearning}.