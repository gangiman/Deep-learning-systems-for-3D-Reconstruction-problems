
\chapter{Introduction}

%video surveillance+
%person re-id+
% originates from Multi-Target Multi-Camera Tracking 
%open world / closed world+
%face : verification/identification
%common aspects: detection , processing, recognition
%deep learning

%retrieval
%fine-grained recognition
%adaptation

%datasets (cut from the papers) + table with reid dataset?
%architectures
%definitions?
%contribution - what is done


\section{Objectives and Motivations}

\subsection{Goals and main results}


%Motivation: RGB-D scanning is here and we want to have a fine-grained understanding of the 3D captures
In the recent years, a wide variety of consumer-grade RGB-D sensors, such as the Intel Real Sense, Microsoft Kinect, depth-sensor enabled smartphones, enabled inexpensive and rapid RGB-D data acquisition. Increasing availability of large, labeled datasets (e.g.,~\cite{chang2017matterport3d,dai2017scannet})  made possible development of deep learning methods for 3D object classification and semantic segmentation. At the same time, acquired 3D data is often incomplete and noisy; while one can identify and segment the objects in the scene, reconstructing high-quality geometry of objects remains a challenging problem.  

An example of the new approach in recent work 
\cite{avetisyan2019scan2cad}, uses a large dataset of clean, labeled geometric shapes
\cite{chang2015shapenet}, for classification/segmentation associating the input point or voxel data with object labels from the dataset, along with adapting geometry to 3D data.  This approach ensures that the output geometry has high quality, and is robust with respect to noise and missing data in the input.  
At the same time, a ``flat'' classification/segmentation approach, with each object in the database corresponding to a separate label and matched to a subpart of the input data corresponding to the whole object, does not scale well as the number of classes grows and often runs into difficulties in the cases of extreme occlusion (only a relatively small part of an object is visible). 
Significant improvements can be achieved by considering object \emph{parts}, or more generally part hierarchies. 
Part-based segmentation of 3D datasets promises to offer a significant improvement both in finding the best matching shape in the dataset, recognizing objects from  highly incomplete data (e.g., from a couple of parts) from  as well as more precise geometry adaptation as well as, potentially, assembly of new shapes out of existing parts yielding a closer match to the input data. 

% large collection of 3D models in database can be reduced to structured representations, 
%objects with occluded sub-parts still can be recognized by parts available in the scan and the rest can be guessed with high probability, using parts, we can reconstruct new objects that are not yet present in the database of shapes. \LA{Whoa: we must either prove this by experiments, or appeal to existing researcg}

%based on different approaches for volumetric information integration, from enhancements of  methods such as volumetric fusion \cite{curless1996volumetric}, to 
%probabilistic  methods, and plethora of methods based on their combinations.

%Compared to computer graphics models manually created by 3D professionals, 3D scans are noisy and incomplete.
% - есть задача восстановления сцен по шумным сканам (всякие слова и ссылки на этот счет были и у нас в статье, и в статьях нисснера про scan2cad)
%Amount of noise and limited resolution of \VI{consumer-grade} consumer grade scanning hardware pose significant challenges for solving this important problem of scene reconstruction. 
%Approaches of reconstruction based on fitting existing 3D assets into scene scans, have shown a lot of promise but still had problems with finding exact models from large database such as ShapeNet \cite{chang2015shapenet}, because of occlusion and lack of spatial context.

% TODO rewrite

%Learning-based approaches are very good at extracting features representative of objects and scenes as a whole, allowing to fill in occluded areas or guess parts affected by noise \cite{dai2017shape,dai2018scancomplete,song2017semantic}. These features are sufficient for scene completion, but they are not as good at recovering geometric primitives like: sharp edges, planar surfaces or borders between sub-parts, resulting in reconstruction quality much poorer than that of 3D content created by humans.

In this work, we focus on the key problem of semantic part segmentation of objects in the scenes, enabling further improvements in  dataset-based reconstruction. 

%\LA{new task: semantic partseg, most parts visible thus apps: nonvisible parts can be inferred, better shape coverage}

In human-made environments, a lot of objects have naturally defined semantic sub-parts, and those sub-parts can, in turn, have their sub-parts, i.e., parts form \emph{hierarchies}.  In our work, we use scene and object representation based on such part hierarchies.  We show how a part-labeled dataset of scanned 3D data suitable for machine learning applications can be constructed, and used to improve the performance of segmentation algorithms. 

%Definitions of sub-parts are based on a set of primitive elements that were manufactured by one formation method or from one material.

%Because of that and the fact that static scenes have other relationships between objects (fixed to each other or in direct surface contact), it's reasonable to suggest a scene description format that possesses a property of hierarchy (e.g., trees or other kinds of graphs).
%We represent scenes as a discrete structures with properties of composability and complex relationships of its parts to compose a whole object and in turn compose a scene from separate objects.
%\LA{usefulness of parts}

%In the last 4 years, the field of computer vision saw an increase in Real-world 3D Scenes datasets acquired using depth sensors and LIDAR's. 
%
%However, due to the limitation of sensors resolution ability, the level of detailization remains the same.
%\DZ{I am not sure this is the reason, and in any case, you need to explain what exactly detailizaiton means in this context}


\todo{merge these paragraphs}

In domain of human-made environments a lot of objects have sub-parts and those sub-parts can in turn have their own sub-parts. Definition of sub-part is often based on a set of primitive elements that were manufactured by one formation method or from one material. Because of that and the fact that static scenes have other relationships between objects (fixed to each other or in direct surface contact), it's reasonable to suggest a scene description format that possess a property of hierarchy (e.g. trees or other subgraphs).
A lot of researchers over the last 20 years came to the same conclusion. A lot of work on that problem was done by Mumford and Zhu in \cite{zhu2006stochastic}.

One of the papers dealt with problem of modeling Images as a hierarchy of super-pixels. \cite{russell2009associative}, or as a tree of geometric primitives (e.g. cylinders, spheres or 3D boxes) \cite{li2017grass}.

% point cloud (PC) turned in to a graph (based on proximity) point-cloud parser network reduces number of nodes and edges and enriches their feature vectors. First part of the decoder network functions similar to Feature Pyramid Network in CNNs, which performs local computations on different scales, followed by "pooling or convolutions" with reduced spatial component and increased feature components, thus leaving only small number of "keypoints" required to outline shapes of objects.



CAD constructor network translates that graph into CAD object (tree with primitives and combination rules). CAD rendered makes a mesh out of that object thus a residual between original PC and Mesh can be calculated.

Proximity Graphs - concept that allows to build a bridge between Point Clouds and Graph Processing. This area of computational geometry has a lot of theoretical results to offer for Deep Learning piplene designer.


\section{Problem of 3D reconstruction}
\subsection{Definitions and examples}
\subsection{Data sources and devices}
\subsection{3D data representations}


\section{Inverse Problem formulation}
\subsection{Overcoming lack of information}

\section{Datasets}
\section{Architectures}
\section{Contributions}

% \chapt{hist}, \chapt{bilinear} and \chapt{gradrev} use  person re-identification architecture of \citep{Yi14} as a baseline method (it is also  described in \sect{intro_architectures}).  \chapt{bilinear} is based on the results of \chapt{hist}: the loss function introduced in \chapt{hist} is used for 
% all the experiments in \chapt{bilinear} as it was demonstrated to show the best performance for person re-identification. 
% The results of \chapt{gradrev} were chronologically the earliest among all the results presented in this work, therefore  methods  from \chapt{hist} and \chapt{bilinear} were not used there. 
% Although the contributions of each of the chapters are independent, they are all parts of building a person re-identification pipeline and can be applied simultaneously. 
% \chapt{wildface} considers domain adaptation for surveillance face recognition and uses the method from \chapt{gradrev} as one of the baselines.
 

