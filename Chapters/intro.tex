% !TEX root = ../Thesis_main.tex
\chapter{Introduction}

\section{Context}
\subsection{Problem of 3D reconstruction}

\subsection{Approaches for solving 3D Reconstruction problem}

Solutions for a reconstruction problem can be grouped in two major groups: 1) Geometric approach - when problem is represented as optimisation of scene state given constrains on projections of scene state to data, 2) Machine Learning approach - inverse model is optimisation.

If $X$ - is scene data, $z$ - is the state of the scene, and $g_i(z)$ - projection function of 3D scene $z$ state to perspective $i$, then to find an optimal 3D reconstruction, one solves this minimisation problem:
\begin{equation}
z_{rec} = \min_z\sum_i|X_i-g_i(z)|_2 .
\end{equation}

This approach only solves problem for once scene and does not provide any semantic information about it, only basic geometric information. 
The second approach is more modern and better fitted for machine learning applications, because instead of optimizing state of the scene, it's optimizes a model that performs computation from input data to some semantic (intrinsic) parameters, and can be described as following optimisation procedure:
\begin{equation}
\min_\theta\sum_i|X_i-g_i(f_\theta(\pi))|_2,\ \ \pi=I_\theta(\{X_i\}_i),\ \ z=f_\theta(\pi),
\end{equation}
where $\pi$ - are scene parameters, $f_\theta(\pi)$ - is a generative model that generates 3D state $z$ and it's function is determined by tunable parameters $\theta$.

In reconstruction process information can be introduced in two possible ways: 1) input signal - data measured by some spatial sensor, 2) by adding a priori knowledge while training the Inverse model or by design choice of reconstruction algprithm. Between the two source exist a fundamental trade-off and detirmination of which is dominant can be quite difficult \cite{tatarchenko2019single}.


\section{Objectives and Motivations}

The goal of this work to improve methods of 3D reconstruction in holistic context using deep learning systems. For efficient applications such as robotics in human environments and mixed reality more advanced machine perception systems are needed. Human perception is a complex system with several properties not all of which are replicated in modern machine perception systems. From cognitive sciences it's known that ability to model environments is one of the most important for perception system, in area of computer vision this is known as an ability to perform \textit{3D reconstruction} of scenes and environments. To solve this problem in a general case requires application of machine learning.

In particular, the development of such holistic deep 3D reconstruction system includes several important tasks:

\begin{itemize}
	\item Capturing scenes, complete with colour and depth data of a sufficient quality,
    \item Recalling objects from large scale database of objects,
    \item Segmenting variety of most common elements from sensor data, such as household objects and architectural components,
    \item Detecting and reconstructing shape and pose of and human bodies.
\end{itemize}

Each of these sub-tasks constitutes a challenge in the context of human perception.

The presence of noise in sensor data (e.g., consumer grade depth cameras) is a serious problem for all downstream sub-tasks, low fidelity of this data causes a considerable compaunding performance drop.

\subsection{3D data representations}

We can describe a 3D object in multiple ways, and codification of it's properties has ramifications about capturing different information about objects and scenes, as well as kinds of models that can regenerate them or computational resources needed to process it.
Each representation has it's own pros and cons. We assume 3D information representation to be positive effective and usefull if it captures more relevant information with less storage requirement (compression), increases signal to noise ratio of data, captures shape and texture properties with minimum trade-off.

Here are some popular examples of 3D data representations:
\begin{enumerate}
	\item Multiple 2D projections - captures surface texture, highly redundant representation if images overlap, also vulnerable to optic illusions.
	\item Voxels - simple, most of the time can be sparse, represents rough volumetric properties vell but losses most of surface properties.
	\item Point Cloud - are sparse in a sense that they don't capture empty space, losing all surface properties besides color and estimated normals and most of volumetric properties.
	\item 2.5D (RGB-D) images are widespread because of cheap measurement devices, capture volumentric depth but succeptable to occlusion of bodies in a scene and records a lot of noise with actual signal.
\end{enumerate}

\subsection{Overcoming lack of information}

%Motivation: RGB-D scanning is here and we want to have a fine-grained understanding of the 3D captures
In the recent years, a wide variety of consumer-grade RGB-D sensors, such as the Intel Real Sense, Microsoft Kinect, depth-sensor enabled smartphones, enabled inexpensive and rapid RGB-D data acquisition. Increasing availability of large, labeled datasets (e.g.,~\cite{chang2017matterport3d,dai2017scannet})  made possible development of deep learning methods for 3D object classification and semantic segmentation. At the same time, acquired 3D data is often incomplete and noisy; while one can identify and segment the objects in the scene, reconstructing high-quality geometry of objects remains a challenging problem.  

An example of the new approach in recent work 
\cite{avetisyan2019scan2cad}, uses a large dataset of clean, labeled geometric shapes
\cite{chang2015shapenet}, for classification/segmentation associating the input point or voxel data with object labels from the dataset, along with adapting geometry to 3D data.  This approach ensures that the output geometry has high quality, and is robust with respect to noise and missing data in the input.  
At the same time, a ``flat'' classification/segmentation approach, with each object in the database corresponding to a separate label and matched to a subpart of the input data corresponding to the whole object, does not scale well as the number of classes grows and often runs into difficulties in the cases of extreme occlusion (only a relatively small part of an object is visible). 
Significant improvements can be achieved by considering object \emph{parts}, or more generally part hierarchies. 
Part-based segmentation of 3D datasets promises to offer a significant improvement both in finding the best matching shape in the dataset, recognizing objects from  highly incomplete data (e.g., from a couple of parts) from  as well as more precise geometry adaptation as well as, potentially, assembly of new shapes out of existing parts yielding a closer match to the input data. 

large collection of 3D models in database can be reduced to structured representations, 
objects with occluded sub-parts still can be recognized by parts available in the scan and the rest can be guessed with high probability, using parts, we can reconstruct new objects that are not yet present in the database of shapes.

Based on different approaches for volumetric information integration, from enhancements of  methods such as volumetric fusion \cite{curless1996volumetric}, to probabilistic  methods, and plethora of methods based on their combinations.

Compared to computer graphics models manually created by 3D professionals, 3D scans are noisy and incomplete.
Amount of noise and limited resolution of consumer-grade scanning hardware pose significant challenges for solving this important problem of scene reconstruction. 
Approaches of reconstruction based on fitting existing 3D assets into scene scans, have shown a lot of promise but still had problems with finding exact models from large database such as ShapeNet \cite{chang2015shapenet}, because of occlusion and lack of spatial context.

Learning-based approaches are very good at extracting features representative of objects and scenes as a whole, allowing to fill in occluded areas or guess parts affected by noise \cite{dai2017shape,dai2018scancomplete,song2017semantic}. These features are sufficient for scene completion, but they are not as good at recovering geometric primitives like: sharp edges, planar surfaces or borders between sub-parts, resulting in reconstruction quality much poorer than that of 3D content created by humans.

In this work, we focus on the key problem of semantic part segmentation of objects in the scenes, enabling further improvements in  dataset-based reconstruction. 
Semantic part-segmentation, can help in these situations, when sufficient number of the object parts is visible model can infer the non-visible parts essentially completing an object in sense of maximum probability conditioned on input data.

In human-made environments, a lot of objects have naturally defined semantic sub-parts, and those sub-parts can, in turn, have their sub-parts, i.e., parts form \emph{hierarchies}.  In our work, we use scene and object representation based on such part hierarchies.  We show how a part-labeled dataset of scanned 3D data suitable for machine learning applications can be constructed, and used to improve the performance of segmentation algorithms. 

Definitions of sub-parts are based on a set of primitive elements that were manufactured by one formation method or from one material.

Because of that and the fact that static scenes have other relationships between objects (fixed to each other or in direct surface contact), it's reasonable to suggest a scene description format that possesses a property of hierarchy (e.g., trees or other kinds of graphs).
Representing scenes as a discrete structures with multiple relationships between nodes. Such relationships like composability of its parts and affordances between whole objects, in turn allowing to compose a scene from separate objects.

\todo{merge these paragraphs}

In domain of human-made environments a lot of objects have sub-parts and those sub-parts can in turn have their own sub-parts. Definition of sub-part is often based on a set of primitive elements that were manufactured by one formation method or from one material. Because of that and the fact that static scenes have other relationships between objects (fixed to each other or in direct surface contact), it's reasonable to suggest a scene description format that possess a property of hierarchy (e.g. trees or other subgraphs).
A lot of researchers over the last 20 years came to the same conclusion. A lot of work on that problem was done by Mumford and Zhu in \cite{zhu2006stochastic}.

One of the papers dealt with problem of modeling Images as a hierarchy of super-pixels. \cite{russell2009associative}, or as a tree of geometric primitives (e.g. cylinders, spheres or 3D boxes) \cite{li2017grass}.

CAD constructor network translates that graph into CAD object (tree with primitives and combination rules). CAD rendered makes a mesh out of that object thus a residual between original PC and Mesh can be calculated.

Proximity Graphs - concept that allows to build a bridge between Point Clouds and Graph Processing. This area of computational geometry has a lot of theoretical results to offer for Deep Learning piplene designer.


\subsection{Definitions and examples}
\subsection{Data sources and devices}

\section{Inverse Graphics Problem formulation}

\cite{rezende2016unsupervised,eslami2016attend,kulkarni2015deep,wu20153d,izadinia2017im2cad}

Inverse graphics approach enables to solve a problem of "real-world" scene understanding through reconstruction of that scene and comparison it to measured data in some form.

Because it's a fairly new method it has some unexplored facets:
\begin{enumerate}
    \item How can we scale to hundreds and thousands of objects with different parameters.
    \item Embedded representations better than procedural generation
    \item Are there format that can have all advantages of CAD models and probabilistic properties that arise from real-world uncertainties.
\end{enumerate}

Central goals of computational perception is to get structured description of scenes from measurements such as photographic images, scans and videos.

Computer Vision as Inverse Graphics is the most rational formulations that could help us achieve this goal.

In the past, it has been hard to directly solve these problems in practice because of computational limitations.

However, it may be right time to take another look at this idea due to significant advances in deep learning for computer vision, probabilistic programming, and computer graphics.

Probabilistic programming - a tool that allows us to implement complex models while keeping ability to perform inference, extend with other probabilistic models by being general-purpose.

Re-formalizing inverse graphics in terms of probabilistic programming and deep learning allow us to solve even more complicated vision problems with off-the-shelf computational technology.

To make this approach scalable, my research can incorporate effective techniques such as: approximate Bayesian computation, differentiable programming for rendering.

Computer Graphics nowadays seems to be improving at a great pace in terms of designing solutions for hard image synthesis problems, but these solutions are usually hand-made and not flexible enough to cover all needs for general-purpose real world object generating, latest advances in generative models can help with that.

\section{Body Shape Reconstruction}

\textit{Body Shape Reconstruction} also known as \textit{Body Shape Regression} is a task 

\noindent 
\textbf{DensePose task.} DensePose-COCO dataset contains a large set of images of people collected ``in the wild'' together with different annotations: (i) bounding boxes, (ii) foreground-background masks, (iii) dense correspondences --- points $p \in S$ of a reference 3D model $S\in\mathbb{R}^3$ of the object associated with triplets $(c, u, v) \in\{1, \ldots, C\} \times[0,1]^{2}$, where $c$ indicates which one of $C$ body parts contains the pixel and $(u,v)$ represents the corresponding location (UV coordinates) in the chart of the part \cite{smpl}.
The DensePose task is then to predict such triplets $(c, u, v)$ for each foreground pixel and every person in the image.
\newline

\noindent \textbf{DensePose R-CNN.} The baseline dense pose prediction model, and all the subsequent works \cite{parsing, uncertainty, monkeys} follow the architecture design of Mask R-CNN \cite{maskrcnn}.

The model is a two-stage: first, it generates class-independent region proposals (boxes), then classifies and refines them using the box/class head. Finally, the DensePose head predicts the body part and UV coordinates for each pixel inside the box. Particularly, the model consists of many different blocks (see Fig.~\ref{fig:scheme}):
\begin{itemize}
    \item \textit{Backbone} to extract features from the image,
    \item \textit{Neck} to integrate features from different feature levels of the backbone to effectively perform multi-scale detection,
    \item \textit{Region proposal network (RPN)} to propose a sparse set of box candidates potentially containing objects,
    \item \textit{Heads} take the features pooled from the bounding box on the corresponding feature level, where the detection occurred, and produce output. The first head is a box/class head, which finally predicts whether the object is present in the box and refines the box coordinates. The second head is the DensePose head that predicts either the pixel belongs to the background or assigns it to one of the 24 DensePose charts, and regresses UV coordinates to each foreground pixel inside the bounding box.
\end{itemize}

\section{Datasets}

Only recently research community started accumuulating sugnifficant amount of algined sensor data to solve large scale 3D reconstruction problems in deep learning context. In last 4 years we saw an explosion of 3D shape databases and 2D-to-3D indoor scene datasets, such as ShapeNet, 2D-3D Semantic, Scannet and Matterport3D datasets. Because accuracy and reacall properties of deep learning models scale with amount and variety of data, number of state-of-the-art models grew as well.

\section{Architectures}

PanopticFusion~\cite{narita2019panopticfusion} is a model that is able to segment large indoor scenes and separate \textit{Stuff} and \textit{Things}.


For outdoor datasets Point Cloud representation of data is more common because of large empty spaces and the way LiDAR sensor collects data. In such setting instance segmentation is nessesary first step for reconstruction~\cite{zhang2020instance}.

\subsection{Multi-view models}

Detailed comparision of multi-view 2D CNN model and 3D volumetric CNN can be found in \cite{qi2016volumetric}.
\subsection{3D convolutional models}

Model generates or processes voxel image with 3D convolutional operation implemented Dense or Sparse.

Semantic information can boost the reconstruction performance because deep learning systems are able to pick up onto consistent signal \cite{jiao2018look,tatarchenko2019single,kendall2018multi}.

\subsection{2D to 3D Projection models}

Model uses camera parameters to compute 3D representation using 2D/2.5D images and projection operation.

\subsection{Point Cloud models}

Neural Network processes a set of points to classify, segment or predict new set of points.

\subsection{Graph-Convolutional models}

Point Clouds turned in to a graph (based on proximity) point-cloud parser network reduces number of nodes and edges and enriches their feature vectors. First part of the decoder network functions similar to Feature Pyramid Network in CNNs, which performs local computations on different scales, followed by "pooling or convolutions" with reduced spatial component and increased feature components, thus leaving only small number of "keypoints" required to outline shapes of objects.


Model processes Geometric Graph where nodes represent points registered on a surface of objects or parts of objects, and has geometric or other information representing edges between points. Layers process activations associated with nodes taking into account the connectivity.

\subsection{Differentiable rendering models}

Rendering operation is implemented in a differentiable way, allowing backpropogation of gradient information from 2D images to meshes and textures of the objects being rendered.

\subsection{Mesh generating models}

Neaural Network layers that can create a mesh by way of processing a template 3D shape with parametracised operation or meshing other 3D shape representation generated by computation from weights and input.

\subsection{Implicit 3D models}
Like inverse graphics network. Model takes 2D images and returns 2D images with 3D properties changed. 

Models like NeRF, PIFU3D, neural network implements a rendering function depending on view angle and other graphics parameters, one model represents one scene and don't generalise for viriety of scene inputs.


\section{Contributions}

% \chapt{hist}, \chapt{bilinear} and \chapt{gradrev} use  person re-identification architecture of \citep{Yi14} as a baseline method (it is also  described in \sect{intro_architectures}).  \chapt{bilinear} is based on the results of \chapt{hist}: the loss function introduced in \chapt{hist} is used for 
% all the experiments in \chapt{bilinear} as it was demonstrated to show the best performance for person re-identification. 
% The results of \chapt{gradrev} were chronologically the earliest among all the results presented in this work, therefore  methods  from \chapt{hist} and \chapt{bilinear} were not used there. 
% Although the contributions of each of the chapters are independent, they are all parts of building a person re-identification pipeline and can be applied simultaneously. 
% \chapt{wildface} considers domain adaptation for surveillance face recognition and uses the method from \chapt{gradrev} as one of the baselines.
 

