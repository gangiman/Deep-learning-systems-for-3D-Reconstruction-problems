\section{Inverse Graphics Problem formulation}

\cite{rezende2016unsupervised,eslami2016attend,kulkarni2015deep,wu20153d,izadinia2017im2cad}

Inverse graphics approach enables to solve a problem of "real-world" scene understanding through reconstruction of that scene and comparison it to measured data in some form.

Because it's a fairly new method it has some unexplored facets:
\begin{enumerate}
    \item How can we scale to hundreds and thousands of objects with different parameters.
    \item Embedded representations better than procedural generation
    \item Are there format that can have all advantages of CAD models and probabilistic properties that arise from real-world uncertainties.
\end{enumerate}

Central goals of computational perception is to get structured description of scenes from measurements such as photographic images, scans and videos.

Computer Vision as Inverse Graphics is the most rational formulations that could help us achieve this goal.

In the past, it has been hard to directly solve these problems in practice because of computational limitations.

However, it may be right time to take another look at this idea due to significant advances in deep learning for computer vision, probabilistic programming, and computer graphics.

Probabilistic programming - a tool that allows us to implement complex models while keeping ability to perform inference, extend with other probabilistic models by being general-purpose.

Re-formalizing inverse graphics in terms of probabilistic programming and deep learning allow us to solve even more complicated vision problems with off-the-shelf computational technology.

To make this approach scalable, my research can incorporate effective techniques such as: approximate Bayesian computation, differentiable programming for rendering.

Computer Graphics nowadays seems to be improving at a great pace in terms of designing solutions for hard image synthesis problems, but these solutions are usually hand-made and not flexible enough to cover all needs for general-purpose real world object generating, latest advances in generative models can help with that.