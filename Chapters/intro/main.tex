% !TEX root = ../../Thesis_main.tex
\chapter{Introduction}

\input Chapters/intro/context.tex

\input Chapters/intro/objectives.tex

\input Chapters/intro/inverse_graphics.tex

\input Chapters/intro/datasets.tex

\section{Contributions}

% \chapt{hist}, \chapt{bilinear} and \chapt{gradrev} use  person re-identification architecture of \citep{Yi14} as a baseline method (it is also  described in \sect{intro_architectures}).  \chapt{bilinear} is based on the results of \chapt{hist}: the loss function introduced in \chapt{hist} is used for 
% all the experiments in \chapt{bilinear} as it was demonstrated to show the best performance for person re-identification. 
% The results of \chapt{gradrev} were chronologically the earliest among all the results presented in this work, therefore  methods  from \chapt{hist} and \chapt{bilinear} were not used there. 
% Although the contributions of each of the chapters are independent, they are all parts of building a person re-identification pipeline and can be applied simultaneously. 
% \chapt{wildface} considers domain adaptation for surveillance face recognition and uses the method from \chapt{gradrev} as one of the baselines.