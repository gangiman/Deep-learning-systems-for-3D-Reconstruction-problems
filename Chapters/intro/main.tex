% !TEX root = ../../Thesis_main.tex
\chapter{Introduction}

\input Chapters/intro/context.tex

\input Chapters/intro/objectives.tex

\input Chapters/intro/inverse_graphics.tex

\input Chapters/intro/datasets.tex

\section{Contributions}


\chapt{retrieval}


\chapt{scan2part}

% , \chapt{bilinear} and \chapt{gradrev} use  person re-identification architecture of \citep{Yi14} as a baseline method (it is also  described in \sect{intro_architectures}).

% \chapt{bilinear} is based on the results of \chapt{hist}: the loss function introduced in \chapt{hist} is used for 
% all the experiments in \chapt{bilinear} as it was demonstrated to show the best performance for person re-identification. 
The results of \chapt{retrieval} were chronologically the earliest among all the results described in this work, therefore methods from \chapt{depth_superresolution} and \chapt{densepose} were not used there.

Although the contributions of each of the chapters are mostly independent, they are all elements of building a holistic scene understanding pipeline and some can be applied in parallel. 
\chapt{scan2cad} considers alternative models for encoder of segmentation architecture and uses sparse 3D CNN models very similar to ones described in \chapt{retrieval}.