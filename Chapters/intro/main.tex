% !TEX root = ../../Thesis_main.tex
\chapter{Introduction}

\input Chapters/intro/context.tex

\input Chapters/intro/objectives.tex

\input Chapters/intro/datasets.tex

\section{Contributions}

\chapt{densepose}, \chapt{retrieval} and \chapt{scan2part} perform reconstruction of closed families of objects, but \chapt{depth_superresolution} describes improvement of RGB-D sensor data by increasing it's resolution with help of prior on data.
All of the chapters use some kind of Convolutional Neural Network architecture adapted for it's specific data domain and computation environment, for example \chapt{depth_superresolution} and \chapt{densepose} describe methods that can be used on end deivces such as smartphones and embedded devices.
The results of \chapt{retrieval} were chronologically the earliest among all the results described in this work, therefore methods from \chapt{depth_superresolution} and \chapt{densepose} were not used there.

Although the contributions of each of the chapters are mostly independent, they are all elements of building a holistic scene understanding pipeline and some can be applied in parallel. 
\chapt{scan2cad} considers alternative models for encoder of segmentation architecture and uses sparse 3D CNN models very similar to ones described in \chapt{retrieval}.