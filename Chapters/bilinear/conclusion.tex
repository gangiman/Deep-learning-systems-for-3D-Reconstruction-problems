\section{Conclusion}
In this chapter we demonstrated an application of new Multi-region Bilinear CNN architecture to the problem of person re-identification.
Having tried different variants of the bilinear architecture, we showed that such architectures give state-of-the-art performance on larger datasets (by the time of submission for publication). %and that keeping some spatial information is important for building better descriptors. 
In particular, Multi-region Bilinear CNN allows to retain some spatial information and to extract more complex features, while increase the number of parameters over the baseline CNN without overfitting. We have demonstrated notable gap between the performance of the Multi-region Bilinear CNN and the performance of the standard CNN \citep{yi2014deep}.  The code for Caffe~\citep{jia2014caffe} is available at: \url{https://github.com/madkn/MultiregionBilinearCNN-ReId}.

\textbf{Acknowledgement:} This research is supported by the Russian Ministry of Science and Education grant RFMEFI57914X0071.