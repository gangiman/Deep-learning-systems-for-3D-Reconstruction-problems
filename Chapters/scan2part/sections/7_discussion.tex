\section{Conclusion}
We introduced a challenging benchmark for individual parts segmentation of objects in real-world, noisy indoor environments and a novel dataset Scan2Part to evaluate on. 
The core of our method is to leverage structural knowledge of objects composition to perform a variety of segmentation tasks in setting with complex geometry, high levels of uncertainty due to noise. To achieve that, we explore the part taxonomies of common objects in indoor scenes, on multiple scales and methods of compressing them for more effective use in machine learning applications. We demonstrated that specific ways of training deep segmentation models like ours Residual-U-Net style architecture are better at capturing inductive biases in structured labels on some parts of a taxonomy but not the others. Further research on relationships between structure of real-world scenes and perception models is required and we hope our benchmark and dataset will accelerate it.  


\section*{Broader Impact}
% Highlight both benefits and risks from your research. 
An obvious benefit of our research is the improved ability of automated systems to interact with complex environments which require perception on different spatial scales. It is also makes possible to extract more value from data gathered using commodity hardware like RGB-D sensors and modern smartphones. Possible risk that arises from application of technology based on our research is people relying on systems with assumption that algorithms have the same affordances as people.
% Highlight uncertainties
Composition properties of objects, parts and materials are diverse and depend on the culture, but less so with emergence of globalised manufacturing.  