\section{Evaluated approaches}
\label{sect:method}


\subsection{Face recognition for the low-quality image domain}
\label{sect:strategies}
In this work we consider and compare two main approaches to face recognition for surveillance data: 1) restoration-based approach and 2) domain adaptation of existing face recognition neural networks. 

We consider two facial image domains: \begin{itemize}
\item domain $S$, $\{X^{S}_{i}| X^{S}_{i} \subseteq S \}_{i=0}^{N_S}$ that includes low-quality facial images $X^{S}_{i}$ captured using surveillance cameras. Usually, there are no identity labels provided, as assigning identity labels is quite challenging and may not even be feasible.
\item domain $I$, $\{(X^{I}_{i}, Y^{I}_{i}) | X^{I}_{i} \subseteq I \}_{i=0}^{N_I}$ that includes facial images $X^{I}_{i}$ harvested from the Internet. These images are usually of higher quality and are taken in good lighting conditions. We assume that the data in this domain are supplied with identity labels $Y_i$.
\end{itemize}  

According to the available labeling, we can consider two different pipelines for building face recognition systems for surveillance data. The first option is the restoration-based approach when we use transform $F^{S \rightarrow I}: S \longrightarrow I$ as a face restoration method and then apply existing recognition neural network $R^{I}$ that is pre-trained on images from the domain $I$. The second option is to use the transform $F^{I \rightarrow S}: I \longrightarrow S$ to transfer the large collections of labeled training data to the target domain of surveillance images. In this scenario, we retrain the existing face recognition networks resulting in the new adapted model $R^{S}$.

More formally, we consider the following two pipelines for face recognition in the domain $S$.  We denote $d^{S}$ and $d^{I}$ the descriptors produced by the domain-specific face recognition models $R^{S}$ and $R^{I}$. These descriptors may be used e.g.\ to identify matching and non-matching faces based on the distances between them. 
 
$F^{S \rightarrow I}: S \longrightarrow I$ and $F^{I \rightarrow S}: I \longrightarrow S$ are the image-level domain transfer mappings. In the restoration-based approach, we train the recognition model $R^{I}$ using labeled data  $\{(X^{I}_{i}, Y^{I}_{i})\}$ where $X^{I}_{i} \subseteq B$. We then test the learned model by computing the descriptors $d^{I}$ after applying the network $F^{S \rightarrow I}$:  $d^{I} = R^{I}(X^{S\rightarrow I}) = R^{I}(F^{S \rightarrow I}(X^{S}))$

In the domain adaptation approach, we train the recognition network $R^{S}$ using labeled data $\{(X^{I \rightarrow A }_{i}, Y^{I}_{i})\}$, where the training examples $X^{I \rightarrow A }_{i} = F^{I \rightarrow S}(X^{I}_i)$, $X^{I}_i \subseteq B$ are obtained by transforming the high-quality images to the low-quality domain using the learned transformation $F^{I \rightarrow S}$. In this case, we apply the learned network directly to the low-quality images by computing and working with their descriptors $d^{S} = R^{S}(X^{S})$. The two approaches are compared below.


%image describing train and test time for both schemes

\subsection{Learning domain transfer mappings}
\label{sect:domain_transfer}
We use the CycleGAN approach~\cite{ZhuPIE17} to simultaneously learn the domain transfer mappings in both directions: $ F^{S \rightarrow I}: S  \longrightarrow I$ (restoration-based approach) and  $F^{I \rightarrow S}: I \longrightarrow S$ (domain adaptation approach). Here we describe the objective functions used for learning the domain transfer architecture.

We use the variant of CycleGAN similar to the one introduced in \cite{LiuNIPS2017} as we found it resulting in more stable and visually more plausible results for our task than the original framework \cite{ZhuPIE17}. Following \cite{LiuNIPS2017}, we decompose the domain transfer mappings into the compositions of encoders and generators: $F^{S \rightarrow I} = G^{I} \odot E^{S} $ and $F^{I \rightarrow S} = G^{S} \odot E^{I} $. Here, the encoders $E^{S}$ and $E^{I}$ transfer input images to the latent space, and generators $G^{I}$ and $G^{S}$ map the input latent codes to the domains $B$ and $A$.
 
For inputs $X^{S} \subseteq S $ and $X^{I} \subseteq I $ the results of their transfer to the opposite domain will be:
\begin{equation}
    X^{S \rightarrow I} = F^{S \rightarrow I}(X^{S}; \theta^{S}_F) = G^{I}(E^{S}(X^{S}))  
\end{equation}
\begin{equation}
    X^{I \rightarrow S} = F^{I \rightarrow S}(X^{I}; \theta^{I}_F) = G^{S}(E^{I}(X^{I}))
\end{equation}

The objective function used by the CycleGAN approach for learning is composed of the two symmetric parts:
\begin{equation}
\mathcal{L} = \mathcal{L}^{S} + \mathcal{L}^{I},
\end{equation} where $\mathcal{L}^{S}$ further decomposes as:
\begin{equation}\label{eq:domain_loss}
     \mathcal{L}^{S} = \mathcal{L}_{\text{GAN}}^S + \lambda_1 \mathcal{L}_{\text{cycle}}^S + \lambda_2 \mathcal{L}_{\text{rec}}^S,
\end{equation}
while $\mathcal{L}^{I}$ has same structure as $\mathcal{L}^{S}$:
\begin{equation}\label{eq:domain_loss2}
     \mathcal{L}^{I} = \mathcal{L}_{\text{GAN}}^I + \lambda_1 \mathcal{L}_{\text{cycle}}^I + \lambda_2 \mathcal{L}_{\text{rec}}^I,
\end{equation}


We now describe each of the terms in \eq{domain_loss}.
The GAN loss serves as the optimization objective for the domain transfer: 

\begin{dmath}
\mathcal{L}_{\text{GAN}}^S = 
    \min_{\theta^{I}_F} \max_{\theta^{S}_D} \mathbb{E}_{x \sim p_{X^{S}}} \log D^{S}(x) +
    \mathbb{E}_{x \sim p_{X^{I}}} \log \big(1 - D^{S}(F^{I \rightarrow S}(x)) \big)\,
\end{dmath}


Here, $D^{S}(X;\theta^{S}_D)$ and $D^{I}(X;\theta^{I}_D)$ are discriminators for the domains $A$ and $B$ that are trained in parallel with the training of the domain transforms.

The other two terms are the so-called cycle consistency loss: 
\begin{equation}
\mathcal{L}_{\text{cycle}}^S = L_1(F^{I \rightarrow S}(F^{S \rightarrow I}(X^{S})), X^{S})  
\end{equation}
and the reconstruction loss:
\begin{equation}
\mathcal{L}_{\text{rec}}^S = L_1(G^{S}(E^{S}(X^{S})), X^{S}) 
\end{equation}
In both terms, $L_1(\cdot,\cdot)$ denotes the $L_1$ distance.

We show the results of transferring the Internet and surveillance images to the other domain in figure \ref{fig:lr_hr_gan_res_ytube_initial_degraded}. While these results look interesting, we do not analyze their visual quality, as we are ultimately interested in the recognition performance rather than obtained visually-convincing images.

\subsection{Learning face recognition models}
\label{sect:face_recognition}

In both scenarios that we compare in this paper, we need to train a face recognition model that turns images into vectorial descriptors. This happens either in domain $I$ (in the face restoration approach) or in domain $S$ (in the domain adaptation approach).

In either case, the goal of the training is to build a deep convolutional network that converts face images to the descriptors, such that matching face images have close descriptors and non-matching face descriptors have dissimilar descriptors. We use the Binomial Deviance loss~\cite{Yi14} to perform such training:
\begin{equation}
\label{eq:bindev}
J_{dev} = \sum_{i,j \in I} w_{i,j} \ln (\exp^{-\alpha (s_{i,j}-\beta) m_{i,j}} +1 )\,. 
\end{equation}
In \eq{bindev}, $I$ is the set of training image indices, and $s_{i,j}$ is the cosine similarity measure between $i$th and $j$th images, $\alpha$ and $\beta$ are the hyper-parameters. Furthermore,
$m_{i,j}$ and $w_{i, j}$ are the training labels and scaling factors for the positive and negative pairs, so that:
\begin{align}
    \label{eq:bindev_weights}
    m_{i, j} = \left\{
	\begin{array}{l l}
		1,  \text{if $(i,j)$ is a positive pair,} & \\ 
	   -C,  \text{if $(i,j)$ is a negative pair,} &
	\end{array}\right.
	\\
     w_{i, j}= \left\{
	\begin{array}{l l}
		\frac {1}{n_1},  \text{if $(i,j)$ is a positive pair,} &\\
		\frac {1}{n_2},  \text{if $(i,j)$ is a negative pair,} &
	\end{array}\right.
\end{align}	
where $n_1$ and $n_2$ are the number of positive and negative pairs in the training set. The parameter $C$ is the negative cost for balancing weights for positive and negative pairs that was introduced in~\cite{Yi14}. We note that the choice of a particular metric learning loss is orthogonal to our study.

Alternatively to the models trained using the setting discussed above, we also consider reusing the VGG face model trained by the authors of~\cite{parkhi2015deep} on the VGG-face dataset.