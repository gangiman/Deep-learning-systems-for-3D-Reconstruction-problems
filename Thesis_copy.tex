%% ----------------------------------------------------------------
%% Thesis.tex -- MAIN FILE (the one that you compile with LaTeX)
%% ----------------------------------------------------------------

% Set up the document
\documentclass[a4paper, 11pt, oneside]{Thesis}  % Use the "Thesis" style, based on the ECS Thesis style by Steve Gunn
\graphicspath{Figures/}  % Location of the graphics files (set up for graphics to be in PDF format)


\usepackage[table,xcdraw]{xcolor}

% Include any extra LaTeX packages required
\usepackage[square, authoryear, comma]{natbib}  % 
%Use the "Natbib" style for the references in the Bibliography
%\usepackage[backend=biber,style=alphabetic,sorting=nyt]{biblatex}
\usepackage{verbatim}  % Needed for the "comment" environment to make LaTeX comments
\usepackage{vector}  % Allows "\bvec{}" and "\buvec{}" for "blackboard" style bold vectors in maths
\hypersetup{urlcolor=blue, colorlinks=true}  % Colours hyperlinks in blue, but this can be distracting if there are many links.


\newcommand{\fig}[1]{Figure~\ref{fig:#1}}
\newcommand{\sect}[1]{Section~\ref{sect:#1}}
\newcommand{\chapt}[1]{Chapter~\ref{chapt:#1}}
\newcommand{\tab}[1]{Table~\ref{tab:#1}}
\newcommand{\alg}[1]{Algorithm~\ref{alg:#1}}
\newcommand{\eq}[1]{(\ref{eq:#1})}

\usepackage{verbatimbox}
\usepackage{todonotes}
\usepackage{gensymb}
\usepackage{enumitem}
%\newcommand{\todo}[1][]{\@latex@warning{TODO #1}\fbox{TODO\dots}}

%\usepackage{graphicx}
%\usepackage{amsmath,amssymb} % define this before the line numbering.
%\usepackage{placeins}
%\usepackage{color}
%\usepackage[width=122mm,left=12mm,paperwidth=146mm,height=193mm,top=12mm,paperheight=217mm]{geometry}
%\usepackage{subfigure}
\usepackage{subcaption}
%\usepackage{xr-hyper}
%\usepackage[colorlinks=true]{hyperref}
%\usepackage{textcomp}

% Tikz-related stuff.
\usepackage{tikz}
\usetikzlibrary{arrows.meta}
\usetikzlibrary{backgrounds}
\usetikzlibrary{calc}
\usetikzlibrary{decorations.markings}
\usetikzlibrary{decorations.text}
\usetikzlibrary{fit}
\usetikzlibrary{positioning}
\usetikzlibrary{shapes.geometric}
\usetikzlibrary{3d}

% PGF-plots-related stuff.
\usepackage{pgfplots}
\usepgfplotslibrary{groupplots}
\pgfplotsset{compat=newest}
\newlength\figureheight
\newlength\figurewidth
% \usepgfplotslibrary{external}
% \tikzexternalize

\usepackage{nomencl}
\makenomenclature
\renewcommand{\nomname}{List of Abbreviations}

\newcommand{\F}{\mathcal{F}}
\newcommand{\Fsub}[1]{\mathbf{F}_{\mbox{\scriptsize #1}}}
\newcommand{\Dsub}[1]{\mathbf{D}_{\mbox{\scriptsize #1}}}
\newcommand{\cFsub}[1]{\mathcal{F}_{\mbox{\scriptsize #1}}}
\newcommand{\mssize}[1]{\mbox{\scriptsize #1}}
\newcommand{\p}{\mathcal{P}}
\newcommand{\K}{\mathcal{K}}
\newcommand{\E}{\mathbf{E}}
\newcommand{\I}{{I}}
\newcommand{\TO}{\tilde{O}}
\newcommand{\D}{\mathbf{D}}
\newcommand{\CH}{\mathcal{H}}
\newcommand{\Y}{\mathbb{Y}}
\newcommand{\BS}{\mathbf{S}}
\newcommand{\conv}{\text{conv}}
\newcommand{\FC}{\text{FC}}
\newcommand{\ra}{\rightarrow}
\newcommand{\B}{\mathcal{B}}
\newcommand{\eps}{\varepsilon}
\newcommand{\x}{\mathbf{x}}
\newcommand{\y}{\mathbf{y}}
\newcommand{\bv}{\mathbf{v}}
\newcommand{\s}{\mathbf{s}}
\newcommand{\R}{\mathbb{R}}
\newcommand{\thet}{\mathbf{\theta}}
\newcommand{\T}{\mathcal{T}}
\newcommand{\Sm}{\mathcal{S}}
\newcommand{\Q}{\mathcal{Q}}
\newcommand{\tH}{{\tilde H}}
\newcommand{\bp}{\mathbf{p}}
\newcommand{\q}{\mathbf{q}}
\newcommand{\m}{\mathbf{m}}
\newcommand{\bF}{\mathbf{F}}
\newcommand{\bR}{\mathbf{R}}
\newcommand{\bt}{\mathbf{t}}
\newcommand{\A}{\mathbf{A}}
\newcommand{\bS}{\mathbf{S}}
\newcommand{\W}{\mathbf{W}}
\newcommand{\bb}{\mathbf{b}}
\newcommand{\bth}{\mathbf{\theta}}
\newcommand{\bbE}{\mathbb{E}}
\newcommand{\Z}{\mathbb{Z}}
\newcommand{\cL}{\mathcal{L}}

\def\mystrut{\rule[-.3\baselineskip]{0pt}{0.5\baselineskip}}

\input{colors}

%% ----------------------------------------------------------------
\begin{document}
\frontmatter      % Begin Roman style (i, ii, iii, iv...) page numbering

% Set up the Title Page
\title  {ALGORITHMS FOR SPEEDING UP CONVOLUTIONAL NEURAL NETWORKS}
\authors  {Vadim Lebedev}
\addresses  {\groupname\\\deptname\\\univname}  % Do not change this here, instead these must be set in the "Thesis.cls" file, please look through it instead
\date       {\today}
\subject    {}
\keywords   {}

\maketitle
%% ----------------------------------------------------------------

\setstretch{1.3}  % It is better to have smaller font and larger line spacing than the other way round

% Define the page headers using the FancyHdr package and set up for one-sided printing
\fancyhead{}  % Clears all page headers and footers
\rhead{\thepage}  % Sets the right side header to show the page number
\lhead{}  % Clears the left side page header

\pagestyle{fancy}  % Finally, use the "fancy" page style to implement the FancyHdr headers

%% ----------------------------------------------------------------
% Declaration Page required for the Thesis, your institution may give you a different text to place here
%\Declaration{

%\addtocontents{toc}{\vspace{1em}}  % Add a gap in the Contents, for aesthetics

%I, AUTHOR NAME, declare that this thesis titled, `THESIS TITLE' and the work presented in it are my own. I confirm that:

%\begin{itemize}
%\item[\tiny{$\blacksquare$}] This work was done wholly or mainly while in candidature for a research degree at this University.

%\item[\tiny{$\blacksquare$}] Where any part of this thesis has previously been submitted for a degree or any other qualification at this University or any other institution, this has been clearly stated.

%\item[\tiny{$\blacksquare$}] Where I have consulted the published work of others, this is always clearly attributed.

%\item[\tiny{$\blacksquare$}] Where I have quoted from the work of others, the source is always given. With the exception of such quotations, this thesis is entirely my own work.

%\item[\tiny{$\blacksquare$}] I have acknowledged all main sources of help.

%\item[\tiny{$\blacksquare$}] Where the thesis is based on work done by myself jointly with others, I have made clear exactly what was done by others and what I have contributed myself.
%\\
%\end{itemize}


%Signed:\\
%\rule[1em]{25em}{0.5pt}  % This prints a line for the signature

%Date:\\
%\rule[1em]{25em}{0.5pt}  % This prints a line to write the date
%}
%\clearpage  % Declaration ended, now start a new page

%% ----------------------------------------------------------------
% The "Funny Quote Page"
%\pagestyle{empty}  % No headers or footers for the following pages

%\null\vfill
% Now comes the "Funny Quote", written in italics
%\textit{``You're just too good to be true \\
%Can't take my eyes off you.''}

%\begin{flushright}
%Bob Crewe, Bob Gaudio
%\end{flushright}

%\vfill\vfill\vfill\vfill\vfill\vfill\null
%\clearpage  % Funny Quote page ended, start a new page
%% ----------------------------------------------------------------

% The Abstract Page
\addtotoc{Abstract}  % Add the "Abstract" page entry to the Contents
\abstract{
\addtocontents{toc}{\vspace{1em}}  % Add a gap in the Contents, for aesthetics
Modern convolutional neural networks excel in many areas, but their practical application is often hindered by their high computational cost. In this work, I provide a systematic review of the literature on the topic of speeding up convolutional neural networks and present three novel methods. The first method uses the low-rank tensor decomposition of convolutional weights to modify neural network architecture. For my second approach, I employ sparsity-inducing regularizer to prepare convolutional layers for structured pruning of the weights and develop the modified implementation of convolution that accelerates proportionally to the achieved sparsity level. The third approach performs fast fine-grained classification with the combination of a light-weight neural network with radial basis function algorithm. My methods fit into existing deep learning frameworks and allow to finetune accelerated models, leading to considerable speedups of modern convolutional architectures.
}

\clearpage  % Abstract ended, start a new page
%% ----------------------------------------------------------------

\setstretch{1.3}  % Reset the line-spacing to 1.3 for body text (if it has changed)

% The Acknowledgements page, for thanking everyone
%\acknowledgements{
%\addtocontents{toc}{\vspace{1em}}  % Add a gap in the Contents, for aesthetics

%The acknowledgements and the people to thank go here, don't forget to include your project advisor\ldots

%}
%\clearpage  % End of the Acknowledgements
%% ----------------------------------------------------------------

\pagestyle{fancy}  %The page style headers have been "empty" all this time, now use the "fancy" headers as defined before to bring them back


%% ----------------------------------------------------------------
%\lhead{\emph{Contents}}  % Set the left side page header to "Contents"
\tableofcontents  % Write out the Table of Contents

%% ----------------------------------------------------------------
%\lhead{\emph{List of Figures}}  % Set the left side page header to "List if Figures"
%\listoffigures  % Write out the List of Figures

%% ----------------------------------------------------------------
%\lhead{\emph{List of Tables}}  % Set the left side page header to "List of Tables"
%\listoftables  % Write out the List of Tables

%% ----------------------------------------------------------------
\setstretch{1.5}  % Set the line spacing to 1.5, this makes the following tables easier to read

%\clearpage  % Start a new page
%\lhead{\emph{Abbreviations}}  % Set the left side page header to "Abbreviations"
%\listofsymbols{ll}  % Include a list of Abbreviations (a table of two columns)
%{
% \textbf{Acronym} & \textbf{W}hat (it) \textbf{S}tands \textbf{F}or \\
%\textbf{LAH} & \textbf{L}ist \textbf{A}bbreviations \textbf{H}ere \\
%
%}

%% ----------------------------------------------------------------
%\clearpage  % Start a new page
%\lhead{\emph{Physical Constants}}  % Set the left side page header to "Physical Constants"
%\listofconstants{lrcl}  % Include a list of Physical Constants (a four column table)
%{
% Constant Name & Symbol & = & Constant Value (with units) \\
%Speed of Light & $c$ & $=$ & $2.997\ 924\ 58\times10^{8}\ \mbox{ms}^{-\mbox{s}}$ (exact)\\

%}

%% ----------------------------------------------------------------
%\clearpage  %Start a new page
%\lhead{\emph{Symbols}}  % Set the left side page header to "Symbols"
%\listofnomenclature{lll}  % Include a list of Symbols (a three column table)
%{
% symbol & name & unit \\
%$a$ & distance & m \\
%$P$ & power & W (Js$^{-1}$) \\
%& & \\ % Gap to separate the Roman symbols from the Greek
%$\omega$ & angular frequency & rads$^{-1}$ \\
%}
%% ----------------------------------------------------------------
% End of the pre-able, contents and lists of things
% Begin the Dedication page

\setstretch{1.3}  % Return the line spacing back to 1.3

%\pagestyle{empty}  % Page style needs to be empty for this page
%\dedicatory{For/Dedicated to/To my\ldots}

\addtocontents{toc}{\vspace{2em}}  % Add a gap in the Contents, for aesthetics


%% ----------------------------------------------------------------
\mainmatter	  % Begin normal, numeric (1,2,3...) page numbering
\pagestyle{fancy}  % Return the page headers back to the "fancy" style

% Include the chapters of the thesis, as separate files
% Just uncomment the lines as you write the chapters



%% ----------------------------------------------------------------
% Now begin the Appendices, including them as separate files

\addtocontents{toc}{\vspace{2em}} % Add a gap in the Contents, for aesthetics

\appendix % Cue to tell LaTeX that the following 'chapters' are Appendices

\input{Appendices/AppendixA}	% Appendix Title

%\input{Appendices/AppendixB} % Appendix Title

%\input{Appendices/AppendixC} % Appendix Title

\addtocontents{toc}{\vspace{2em}}  % Add a gap in the Contents, for aesthetics
\backmatter

%% ----------------------------------------------------------------
\label{Bibliography}
\lhead{\emph{Bibliography}}  % Change the left side page header to "Bibliography"
\bibliographystyle{plainnat}  % Use the "unsrtnat" BibTeX style for formatting the Bibliography
\bibliography{Bibliography}  % The references (bibliography) information are stored in the file named "Bibliography.bib"

\end{document}  % The End
%% ---------------------------------------------------------------- 